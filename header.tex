%% PDFメタデータの文字化け防止
% https://blog.miz-ar.info/2015/09/latex-hyperref-tips/
% https://tex.stackexchange.com/questions/24445/hyperref-lualatex-and-unicode-bookmarks-issue-garbled-page-numbers-in-ar-for-l
\hypersetup{%
  pdfencoding=auto
}

%% Fonts
\usefonttheme[onlymath]{serif}
\usepackage[T1]{fontenc}
\usepackage{textcomp}
%\usepackage{arev}
\usepackage[scale=1.0]{tgheros}  % San serif font
\usepackage[scaled]{beramono}    % Monospace font

%% Japanese font
\usepackage{luatexja-otf}
\usepackage[match,deluxe,expert,haranoaji,nfssonly]{luatexja-preset} % Notoフォント使用
\renewcommand{\kanjifamilydefault}{\gtdefault}

%%
\setbeamerfont{title}{size=\huge, series=\bfseries}
\setbeamerfont{frametitle}{size=\Large, series=\bfseries}

%% https://tex.stackexchange.com/questions/62202/change-background-colour-of-verbatim-environment
\let\oldv\verbatim
\let\oldendv\endverbatim
\def\verbatim{\par\setbox0\vbox\bgroup\oldv}
\def\endverbatim{\oldendv\egroup\fboxsep0pt \noindent\colorbox[gray]{0.95}{\usebox0}\par}

%% https://stackoverflow.com/questions/38323331/code-chunk-font-size-in-beamer-with-knitr-and-latex
%% change fontsize of R code
\let\oldShaded\Shaded
\let\endoldShaded\endShaded 
\renewenvironment{Shaded}{\footnotesize\oldShaded}{\endoldShaded}

\usepackage{listings}
\lstset{%
  frame = shadow,
  backgroundcolor = {\color[gray]{.95}},
  basicstyle = {\small\ttfamily},
  breaklines = true,
  upquote = true
}
%%%%%%%%%%%%%%%
\usepackage{booktabs}
