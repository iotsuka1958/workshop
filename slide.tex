% Options for packages loaded elsewhere
\PassOptionsToPackage{unicode}{hyperref}
\PassOptionsToPackage{hyphens}{url}
%
\documentclass[
  ignorenonframetext,
]{beamer}
\usepackage{pgfpages}
\setbeamertemplate{caption}[numbered]
\setbeamertemplate{caption label separator}{: }
\setbeamercolor{caption name}{fg=normal text.fg}
\beamertemplatenavigationsymbolsempty
% Prevent slide breaks in the middle of a paragraph
\widowpenalties 1 10000
\raggedbottom
\setbeamertemplate{part page}{
  \centering
  \begin{beamercolorbox}[sep=16pt,center]{part title}
    \usebeamerfont{part title}\insertpart\par
  \end{beamercolorbox}
}
\setbeamertemplate{section page}{
  \centering
  \begin{beamercolorbox}[sep=12pt,center]{part title}
    \usebeamerfont{section title}\insertsection\par
  \end{beamercolorbox}
}
\setbeamertemplate{subsection page}{
  \centering
  \begin{beamercolorbox}[sep=8pt,center]{part title}
    \usebeamerfont{subsection title}\insertsubsection\par
  \end{beamercolorbox}
}
\AtBeginPart{
  \frame{\partpage}
}
\AtBeginSection{
  \ifbibliography
  \else
    \frame{\sectionpage}
  \fi
}
\AtBeginSubsection{
  \frame{\subsectionpage}
}
\usepackage{amsmath,amssymb}
\usepackage{lmodern}
\usepackage{iftex}
\ifPDFTeX
  \usepackage[T1]{fontenc}
  \usepackage[utf8]{inputenc}
  \usepackage{textcomp} % provide euro and other symbols
\else % if luatex or xetex
  \usepackage{unicode-math}
  \defaultfontfeatures{Scale=MatchLowercase}
  \defaultfontfeatures[\rmfamily]{Ligatures=TeX,Scale=1}
\fi
% Use upquote if available, for straight quotes in verbatim environments
\IfFileExists{upquote.sty}{\usepackage{upquote}}{}
\IfFileExists{microtype.sty}{% use microtype if available
  \usepackage[]{microtype}
  \UseMicrotypeSet[protrusion]{basicmath} % disable protrusion for tt fonts
}{}
\makeatletter
\@ifundefined{KOMAClassName}{% if non-KOMA class
  \IfFileExists{parskip.sty}{%
    \usepackage{parskip}
  }{% else
    \setlength{\parindent}{0pt}
    \setlength{\parskip}{6pt plus 2pt minus 1pt}}
}{% if KOMA class
  \KOMAoptions{parskip=half}}
\makeatother
\usepackage{xcolor}
\newif\ifbibliography
\usepackage{color}
\usepackage{fancyvrb}
\newcommand{\VerbBar}{|}
\newcommand{\VERB}{\Verb[commandchars=\\\{\}]}
\DefineVerbatimEnvironment{Highlighting}{Verbatim}{commandchars=\\\{\}}
% Add ',fontsize=\small' for more characters per line
\usepackage{framed}
\definecolor{shadecolor}{RGB}{248,248,248}
\newenvironment{Shaded}{\begin{snugshade}}{\end{snugshade}}
\newcommand{\AlertTok}[1]{\textcolor[rgb]{0.94,0.16,0.16}{#1}}
\newcommand{\AnnotationTok}[1]{\textcolor[rgb]{0.56,0.35,0.01}{\textbf{\textit{#1}}}}
\newcommand{\AttributeTok}[1]{\textcolor[rgb]{0.77,0.63,0.00}{#1}}
\newcommand{\BaseNTok}[1]{\textcolor[rgb]{0.00,0.00,0.81}{#1}}
\newcommand{\BuiltInTok}[1]{#1}
\newcommand{\CharTok}[1]{\textcolor[rgb]{0.31,0.60,0.02}{#1}}
\newcommand{\CommentTok}[1]{\textcolor[rgb]{0.56,0.35,0.01}{\textit{#1}}}
\newcommand{\CommentVarTok}[1]{\textcolor[rgb]{0.56,0.35,0.01}{\textbf{\textit{#1}}}}
\newcommand{\ConstantTok}[1]{\textcolor[rgb]{0.00,0.00,0.00}{#1}}
\newcommand{\ControlFlowTok}[1]{\textcolor[rgb]{0.13,0.29,0.53}{\textbf{#1}}}
\newcommand{\DataTypeTok}[1]{\textcolor[rgb]{0.13,0.29,0.53}{#1}}
\newcommand{\DecValTok}[1]{\textcolor[rgb]{0.00,0.00,0.81}{#1}}
\newcommand{\DocumentationTok}[1]{\textcolor[rgb]{0.56,0.35,0.01}{\textbf{\textit{#1}}}}
\newcommand{\ErrorTok}[1]{\textcolor[rgb]{0.64,0.00,0.00}{\textbf{#1}}}
\newcommand{\ExtensionTok}[1]{#1}
\newcommand{\FloatTok}[1]{\textcolor[rgb]{0.00,0.00,0.81}{#1}}
\newcommand{\FunctionTok}[1]{\textcolor[rgb]{0.00,0.00,0.00}{#1}}
\newcommand{\ImportTok}[1]{#1}
\newcommand{\InformationTok}[1]{\textcolor[rgb]{0.56,0.35,0.01}{\textbf{\textit{#1}}}}
\newcommand{\KeywordTok}[1]{\textcolor[rgb]{0.13,0.29,0.53}{\textbf{#1}}}
\newcommand{\NormalTok}[1]{#1}
\newcommand{\OperatorTok}[1]{\textcolor[rgb]{0.81,0.36,0.00}{\textbf{#1}}}
\newcommand{\OtherTok}[1]{\textcolor[rgb]{0.56,0.35,0.01}{#1}}
\newcommand{\PreprocessorTok}[1]{\textcolor[rgb]{0.56,0.35,0.01}{\textit{#1}}}
\newcommand{\RegionMarkerTok}[1]{#1}
\newcommand{\SpecialCharTok}[1]{\textcolor[rgb]{0.00,0.00,0.00}{#1}}
\newcommand{\SpecialStringTok}[1]{\textcolor[rgb]{0.31,0.60,0.02}{#1}}
\newcommand{\StringTok}[1]{\textcolor[rgb]{0.31,0.60,0.02}{#1}}
\newcommand{\VariableTok}[1]{\textcolor[rgb]{0.00,0.00,0.00}{#1}}
\newcommand{\VerbatimStringTok}[1]{\textcolor[rgb]{0.31,0.60,0.02}{#1}}
\newcommand{\WarningTok}[1]{\textcolor[rgb]{0.56,0.35,0.01}{\textbf{\textit{#1}}}}
\usepackage{longtable,booktabs,array}
\usepackage{calc} % for calculating minipage widths
\usepackage{caption}
% Make caption package work with longtable
\makeatletter
\def\fnum@table{\tablename~\thetable}
\makeatother
\usepackage{graphicx}
\makeatletter
\def\maxwidth{\ifdim\Gin@nat@width>\linewidth\linewidth\else\Gin@nat@width\fi}
\def\maxheight{\ifdim\Gin@nat@height>\textheight\textheight\else\Gin@nat@height\fi}
\makeatother
% Scale images if necessary, so that they will not overflow the page
% margins by default, and it is still possible to overwrite the defaults
% using explicit options in \includegraphics[width, height, ...]{}
\setkeys{Gin}{width=\maxwidth,height=\maxheight,keepaspectratio}
% Set default figure placement to htbp
\makeatletter
\def\fps@figure{htbp}
\makeatother
\setlength{\emergencystretch}{3em} % prevent overfull lines
\providecommand{\tightlist}{%
  \setlength{\itemsep}{0pt}\setlength{\parskip}{0pt}}
\setcounter{secnumdepth}{-\maxdimen} % remove section numbering
%% PDFメタデータの文字化け防止
% https://blog.miz-ar.info/2015/09/latex-hyperref-tips/
% https://tex.stackexchange.com/questions/24445/hyperref-lualatex-and-unicode-bookmarks-issue-garbled-page-numbers-in-ar-for-l
\hypersetup{%
  pdfencoding=auto
}

%% Fonts
\usefonttheme[onlymath]{serif}
\usepackage[T1]{fontenc}
\usepackage{textcomp}
%\usepackage{arev}
\usepackage[scale=1.0]{tgheros}  % San serif font
\usepackage[scaled]{beramono}    % Monospace font

%% Japanese font
\usepackage{luatexja-otf}
\usepackage[match,deluxe,expert,haranoaji,nfssonly]{luatexja-preset} % Notoフォント使用
\renewcommand{\kanjifamilydefault}{\gtdefault}

%%
\setbeamerfont{title}{size=\huge, series=\bfseries}
\setbeamerfont{frametitle}{size=\Large, series=\bfseries}

%% https://tex.stackexchange.com/questions/62202/change-background-colour-of-verbatim-environment
\let\oldv\verbatim
\let\oldendv\endverbatim
\def\verbatim{\par\setbox0\vbox\bgroup\oldv}
\def\endverbatim{\oldendv\egroup\fboxsep0pt \noindent\colorbox[gray]{0.95}{\usebox0}\par}

%% https://stackoverflow.com/questions/38323331/code-chunk-font-size-in-beamer-with-knitr-and-latex
%% change fontsize of R code
\let\oldShaded\Shaded
\let\endoldShaded\endShaded 
\renewenvironment{Shaded}{\footnotesize\oldShaded}{\endoldShaded}

\usepackage{listings}
\lstset{%
  frame = shadow,
  backgroundcolor = {\color[gray]{.95}},
  basicstyle = {\small\ttfamily},
  breaklines = true,
  upquote = true
}
%%%%%%%%%%%%%%%
\usepackage{booktabs}
\ifLuaTeX
  \usepackage{selnolig}  % disable illegal ligatures
\fi
\IfFileExists{bookmark.sty}{\usepackage{bookmark}}{\usepackage{hyperref}}
\IfFileExists{xurl.sty}{\usepackage{xurl}}{} % add URL line breaks if available
\urlstyle{same} % disable monospaced font for URLs
\hypersetup{
  pdfauthor={iotsuka},
  hidelinks,
  pdfcreator={LaTeX via pandoc}}

\title{oyoyo\\
hogehoge研修}
\author{iotsuka}
\date{2022-06-08}

\begin{document}
\frame{\titlepage}

\begin{frame}{data}
\protect\hypertarget{data}{}
\begin{itemize}[<+->]
\item
  4つのデータ(data\_1 ~ data\_4)があります
\item
  それぞれのデータは、11組のxとyから構成されています
\end{itemize}
\end{frame}

\begin{frame}{data}
\protect\hypertarget{data-1}{}
\begin{tabular}{rr}
\multicolumn{2}{l}{data\_1}\\
\toprule
x & y\\
\midrule
10 & 8.04\\
8 & 6.95\\
13 & 7.58\\
9 & 8.81\\
11 & 8.33\\
14 & 9.96\\
6 & 7.24\\
4 & 4.26\\
12 & 10.84\\
7 & 4.82\\
5 & 5.68\\
\bottomrule
\end{tabular}\hfill
\begin{tabular}{rr}
\multicolumn{2}{l}{data\_2}\\
\toprule
x & y\\
\midrule
10 & 9.14\\
8 & 8.14\\
13 & 8.74\\
9 & 8.77\\
11 & 9.26\\
14 & 8.10\\
6 & 6.13\\
4 & 3.10\\
12 & 9.13\\
7 & 7.26\\
5 & 4.74\\
\bottomrule
\end{tabular}\hfill
\begin{tabular}{rr}
\multicolumn{2}{l}{data\_3}\\
\toprule
x & y\\
\midrule
10 & 7.46\\
8 & 6.77\\
13 & 12.74\\
9 & 7.11\\
11 & 7.81\\
14 & 8.84\\
6 & 6.08\\
4 & 5.39\\
12 & 8.15\\
7 & 6.42\\
5 & 5.73\\
\bottomrule
\end{tabular}\hfill
\begin{tabular}{rr}
\multicolumn{2}{l}{data\_4}\\
\toprule
x & y\\
\midrule
8 & 6.58\\
8 & 5.76\\
8 & 7.71\\
8 & 8.84\\
8 & 8.47\\
8 & 7.04\\
8 & 5.25\\
19 & 12.50\\
8 & 5.56\\
8 & 7.91\\
8 & 6.89\\
\bottomrule
\end{tabular}
\end{frame}

\begin{frame}{data}
\protect\hypertarget{data-2}{}
それぞれどういうデータなんだろう

\begin{itemize}[<+->]
\tightlist
\item
  表だけ見てもよくわからない
\end{itemize}
\end{frame}

\begin{frame}{data}
\protect\hypertarget{data-3}{}
それぞれどういうデータなんだろう

\begin{itemize}[<+->]
\tightlist
\item
  xの平均
\item
  xの分散
\item
  yの平均
\item
  yの分散
\item
  xとyの相関係数
\item
  回帰直線の式
\end{itemize}
\end{frame}

\begin{frame}{summarise}
\protect\hypertarget{summarise}{}
\begin{longtable}[]{@{}lrrrrr@{}}
\toprule()
data & x\_mean & x\_var & y\_mean & y\_var & cor \\
\midrule()
\endhead
data\_1 & 9 & 11 & 7.500909 & 4.127269 & 0.8164205 \\
data\_2 & 9 & 11 & 7.500909 & 4.127629 & 0.8162365 \\
data\_3 & 9 & 11 & 7.500000 & 4.122620 & 0.8162867 \\
data\_4 & 9 & 11 & 7.500909 & 4.123249 & 0.8165214 \\
\bottomrule()
\end{longtable}

\begin{itemize}
\tightlist
\item
  要約統計量はよく似ている
\end{itemize}
\end{frame}

\begin{frame}[fragile]{visualization}
\protect\hypertarget{visualization}{}
\begin{figure}
\centering
\includegraphics{slide_files/figure-beamer/unnamed-chunk-4-1.pdf}
\caption{\texttt{ggplot2} によるグラフ}
\end{figure}
\end{frame}

\begin{frame}[fragile]{visualization}
\protect\hypertarget{visualization-1}{}
\begin{figure}
\centering
\includegraphics{slide_files/figure-beamer/unnamed-chunk-5-1.pdf}
\caption{\texttt{ggplot2} によるグラフ}
\end{figure}
\end{frame}

\begin{frame}{lesson}
\protect\hypertarget{lesson}{}
表でみてもよくわからない

平均値とか分散を計算してもよくわからない

図示がだいじ
\end{frame}

\begin{frame}{research}
\protect\hypertarget{research}{}
\LARGE

\begin{itemize}[<+->]
\tightlist
\item
  \textbullet ある町
\item
  \textbullet\hspace{1pt} 4つの学校
\item
  \textbullet 同じテスト
\end{itemize}
\end{frame}

\begin{frame}{research}
\protect\hypertarget{research-1}{}
\LARGE

\begin{itemize}[<+->]
\tightlist
\item
  生徒対象に調査
\item
   \textbullet テスト勉強にあてた時間
\item
   \textbullet 実際の点数
\end{itemize}
\end{frame}

\begin{frame}{hypothethis}
\protect\hypertarget{hypothethis}{}
\LARGE

勉強時間と点数に関係があるでしょうか

あるとしたらどういう関係が\ldots
\end{frame}

\begin{frame}{plot}
\protect\hypertarget{plot}{}
\includegraphics{slide_files/figure-beamer/unnamed-chunk-8-1.pdf}
\end{frame}

\begin{frame}{plot}
\protect\hypertarget{plot-1}{}
\includegraphics{slide_files/figure-beamer/unnamed-chunk-9-1.pdf}
\end{frame}

\begin{frame}{plot}
\protect\hypertarget{plot-2}{}
\includegraphics[width=\textwidth]{./letsnotsee.jpg}
\end{frame}

\begin{frame}{plot}
\protect\hypertarget{plot-3}{}
\includegraphics{slide_files/figure-beamer/unnamed-chunk-10-1.pdf}
\end{frame}

\begin{frame}{plot}
\protect\hypertarget{plot-4}{}
\includegraphics{slide_files/figure-beamer/unnamed-chunk-11-1.pdf}
\end{frame}

\begin{frame}{plot}
\protect\hypertarget{plot-5}{}
\includegraphics{slide_files/figure-beamer/unnamed-chunk-12-1.pdf}
\end{frame}

\begin{frame}{plot}
\protect\hypertarget{plot-6}{}
\includegraphics{slide_files/figure-beamer/unnamed-chunk-13-1.pdf}
\end{frame}

\begin{frame}{plot}
\protect\hypertarget{plot-7}{}
\includegraphics{slide_files/figure-beamer/unnamed-chunk-14-1.pdf}
\end{frame}

\begin{frame}{Simpson's Paradox}
\protect\hypertarget{simpsons-paradox}{}
\large

\begin{itemize}[<+->]
\tightlist
\item
  \textbullet{}\hspace{2pt}全体で見ると「勉強すればするほど成績がさがる」
\item
  \textbullet{}\hspace{2pt}学校ごとに見ると「勉強すればするほど成績があがる」
\end{itemize}

\normalsize
\end{frame}

\begin{frame}{quiz:How many millions are in a trillion?}
\protect\hypertarget{quizhow-many-millions-are-in-a-trillion}{}
\includegraphics{slide_files/figure-beamer/unnamed-chunk-15-1.pdf}
\end{frame}

\begin{frame}{quiz:How many millions are in a trillion?}
\protect\hypertarget{quizhow-many-millions-are-in-a-trillion-1}{}
\LARGE

1兆は100万の何倍でしょう?

\begin{tabular}{rr}\toprule
回答&率\\\midrule
千倍&18\%\\
万倍&12\%\\
10万倍&21\%\\
100万倍&21\%\\
1000万倍&17\%\\
わからない&12\%\\\bottomrule
\end{tabular}
\end{frame}

\begin{frame}{pie chart}
\protect\hypertarget{pie-chart}{}
\includegraphics[height=8cm]{slide_files/figure-beamer/unnamed-chunk-17-1}
\end{frame}

\begin{frame}{substitute for a pie chart}
\protect\hypertarget{substitute-for-a-pie-chart}{}
\includegraphics{slide_files/figure-beamer/unnamed-chunk-18-1.pdf}
\end{frame}

\begin{frame}{substitute for a pie chart}
\protect\hypertarget{substitute-for-a-pie-chart-1}{}
\includegraphics{slide_files/figure-beamer/unnamed-chunk-19-1.pdf}
\end{frame}

\begin{frame}[fragile]{substitute for a pie chart}
\protect\hypertarget{substitute-for-a-pie-chart-2}{}
\begin{verbatim}
## Warning: Using `size` aesthetic for lines was deprecated in ggplot2 3.4.0.
## i Please use `linewidth` instead.
\end{verbatim}

\includegraphics{slide_files/figure-beamer/unnamed-chunk-20-1.pdf}
\end{frame}

\begin{frame}{quiz:How many millions are in a trillion?}
\protect\hypertarget{quizhow-many-millions-are-in-a-trillion-2}{}
\includegraphics{slide_files/figure-beamer/unnamed-chunk-21-1.pdf}
\end{frame}

\begin{frame}{quiz:whaddayathink?}
\protect\hypertarget{quizwhaddayathink}{}
\Huge

有病率 0.1\%
\end{frame}

\begin{frame}{quiz:whaddayathink?}
\protect\hypertarget{quizwhaddayathink-1}{}
\Large

その検査は\ldots{}
\end{frame}

\begin{frame}{quiz:whaddayathink?}
\protect\hypertarget{quizwhaddayathink-2}{}
\Large

\begin{itemize}[<+->]
\tightlist
\item
  \textbullet{}\hspace{2pt}病気のとき陽性になる確率は99.0\%
\item
  \textbullet{}\hspace{2pt}病気でないとき陰性になる確率は97.0\%
\end{itemize}
\end{frame}

\begin{frame}{quiz:whaddayathink?}
\protect\hypertarget{quizwhaddayathink-3}{}
\Huge

\begin{itemize}[<+->]
\item
  その検査を受けました
\item
  → 陽性
\end{itemize}
\end{frame}

\begin{frame}{quiz:whaddayathink?}
\protect\hypertarget{quizwhaddayathink-4}{}
\Huge

実際に病気にかかっている確率は??
\end{frame}

\begin{frame}[fragile]{Slide with R Output}
\protect\hypertarget{slide-with-r-output}{}
\begin{Shaded}
\begin{Highlighting}[]
\FunctionTok{summary}\NormalTok{(cars)}
\end{Highlighting}
\end{Shaded}

\begin{verbatim}
##      speed           dist       
##  Min.   : 4.0   Min.   :  2.00  
##  1st Qu.:12.0   1st Qu.: 26.00  
##  Median :15.0   Median : 36.00  
##  Mean   :15.4   Mean   : 42.98  
##  3rd Qu.:19.0   3rd Qu.: 56.00  
##  Max.   :25.0   Max.   :120.00
\end{verbatim}
\end{frame}

\begin{frame}{文書・資料}
\protect\hypertarget{ux6587ux66f8ux8cc7ux6599}{}
To do our work, we all have to read a mass of papers. Nearly all of them
are far too long. This wastes time, while energy has to be spent in
looking for the essential points.

I ask my colleagues and their staffs to see to it that their Reports are
shorter.
\end{frame}

\begin{frame}{文書・資料}
\protect\hypertarget{ux6587ux66f8ux8cc7ux6599-1}{}
\begin{itemize}[<+->]
\tightlist
\item
  \textbullet{}\hspace{2pt} The aim should be Reports which set out the
  main points in a series of short, crisp paragraphs.
\item
  \textbullet{}\hspace{2pt} If a Report relies on detailed analysis of
  some complicated factors, or on statistics, these should be set out in
  an Appendix.
\item
  \textbullet{}\hspace{2pt} Often the occasion is best met by submitting
  not a full-dress Report, but an Aide-memoire consisting of headings
  only, which can be expanded orally if needed.
\item
  \textbullet\hspace{2pt}Let us have an end of such phrases as these:
  ``It is also of importance to bear in mind the following
  considerations\ldots{}'', or ``Consideration should be given to the
  possibility of carrying into effect\ldots{}'' Most of these woolly
  phrases are mere padding, which can be left out altogether, or
  replaced by a single word. Let us not shrink from using the short
  expressive phrase, even if it is conversational.
\end{itemize}
\end{frame}

\begin{frame}{文書・資料}
\protect\hypertarget{ux6587ux66f8ux8cc7ux6599-2}{}
Reports drawn up on the lines I propose may at first seem rough as
compared with the flat surface of officialese jargon. But the saving in
time will be great, while the discipline of setting out the real points
concisely will prove an aid to clearer thinking.
\end{frame}

\begin{frame}{Slide with Plot}
\protect\hypertarget{slide-with-plot}{}
\includegraphics{slide_files/figure-beamer/pressure-1.pdf}
\end{frame}

\begin{frame}{tohoho}
\protect\hypertarget{tohoho}{}
\includegraphics{slide_files/figure-beamer/unnamed-chunk-25-1.pdf}
\end{frame}

\begin{frame}{tohoho}
\protect\hypertarget{tohoho-1}{}
\includegraphics{slide_files/figure-beamer/unnamed-chunk-26-1.pdf}
\end{frame}

\begin{frame}[fragile]{tohoho}
\protect\hypertarget{tohoho-2}{}
\begin{figure}
\centering
\includegraphics{slide_files/figure-beamer/plot-sample-1.pdf}
\caption{\texttt{ggplot2} によるグラフ}
\end{figure}
\end{frame}

\end{document}
