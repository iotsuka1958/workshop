\documentclass[uplatex,jis2004,dvipdfmx,12pt]{jsarticle}
\usepackage[hiresbb]{graphicx}
\usepackage[deluxe]{otf}
\usepackage{amsmath}
\usepackage{longtable}%%%hyperrefより先に読むのがだいじらしい
\usepackage{okumacro}
%%%%%%%%%%%%%%%%%
\PassOptionsToPackage{dvipsnames,table,dvipdfmx}{xcolor}%tikzパッケージよりも前に読み込みます。
\usepackage[bww,arrow= red]{callouts}
\usepackage{tikz}
\usepackage{pxpgfmark} % remember picture を可能にする
\usepackage[yyyymmdd]{datetime}
\usepackage{array,colortbl,xcolor}
\usepackage{enumitem}%%[label=\textbf{\arabic*}]
\usepackage{arydshln}
\usepackage{amsmath}
\usepackage{amssymb}
\usepackage{niceframe}
\usepackage{multicol}
\usepackage{bxwareki}%%%和暦
\usepackage{mymacros}
%%%%%%%%%%%%%%%%%%%%%%
\usepackage{ascmac}
\usepackage{booktabs}%%%%%tabularの横線の改良\toprule\midrule\bottomrule
%%%%%%%%%%%%%%%%%%%%%%%%%%%
\usepackage{qtree}
%%%%%%%%%%%%%%%%%%%%%%%%%%
\begin{document}
本日は、限られた時間ですので、大きく2つ話をさせてください。

1つは「データ」の話。
もうひとつは「文書」の話です。


\section{データ}
では、「データ」から。

とある予算関係の会議でのことです。

「この事業ではこれまでどんな成果がありましたか。今後はどんな成果が見込ま
れますか」と問われた課の職員がこう答えました。
「子供の目が輝き、笑顔がこぼれました。この事業によりもっと明るい笑顔が期
待できます」

これは実に美しい話だし、
うそ偽りなくきらめく笑顔がこぼれたとおもうのですが、
これで財布を握っている財政当局が予算を認めてくれるかどうかは、
また別のお話です。

ここはどうしてもデータの裏付けが必要な場面です。

みなさんの思いは重要ですが、
思いだけでは仕事は前に進まない。

行政職員としては、自分の思いを事業化して、
社会のためになにかしたいというところはみな共通かなとおもいますが、
前に進むためには、データは避けてとおれません。

Evidence-Based Policy-Making,
make a policy based upon evidenceからできたことば、
「エビデンスに基づいて政策を立案する」ということで、
エビデンスは必ずしも数字である必要はないわけですが、
数字がだせればそれがいいわけです。

で、いろいろなデータをどう読み込むのかということが問題になってきます。

各自で勉強するといえば、それで終わってしまいますが、
きょうは問題をやってみようとおもいます。

実は、先日、ある病気の検査を受けました。

人口の0.8\%の人が、この病気にかかっていることが知られています。

有病率 $=$ 0.8\%



この検査では、
病気にかかっている人が受けると陽性になる確率は90\%。

この確率のことを「感度」といいます。

また、
この病気にかかっていない人が受けると陰性になる確率は93\%。
この確率を「特異度」といいます。

病気の場合、90\%が陽性になるということは、残りの10\%は病気を見逃されるというこ
とになりますし、
病気でない場合93\%で陰性になるということは、残りの7\%は病気でないのに陽
性になって心配することになります。

完璧な検査というものが理想ですが、なかなかそうはいきません。
完璧な検査は感度も特異度もともに100\%ですが、
通常これらはトレードオフの関係。
片方がよくなるともう一方は悪くなります。

きょくたんなことをいいます。
感度をあげたければ、かたっぱしから陽性にすればいい。
そうすれば見落としはなくなりますが、病気でないのに陽性になる人がたくさん
でてくる、
しなくてもいい心配をすることになります。

その逆もしかり。
特異度をあげたければ「疑わしきは罰せず」というか、
基本陰性にすれば病気でないのに陽性という人は少なくなります。
そのかわり病気の人も見逃されて陰性になってしまう。
なかには手遅れになってしまう人もでてくるでしょう。

一般的に言えばそういうことになります。

わたしが受けた検査は、
感度90\%、特異度93\%ですから、
完璧ではないけれど、まあまあ優秀な検査。

あ、それでわたしの検査結果は陽性。

ここで問題。
わたしが本当にこの病気にかかっている確率はどれくらいでしょうか。

これ時間があればいっしょに計算したいところですが、
まあ、みなさんのカンでいうとどれくらいでしょうか。

指名はしませんのでリラックスしていただければ。

1,000人いるとします。

そうすると有病率0.8\%ですから、
病気の人は8人。


感度は90\%なので、
陽性になるのは$8*.9=7.2$人。


いっぽう病気でない人は、
1000-8=992人。

特異度が93\%ですから、
陰性になるのは
992*.93=922.56人。

7\%の人が陽性になってしまうので
992*.07=69.44人が陽性。

そうすると
陽性になるのは7.2人+69.44人=76.64人。

実際に病気にかかっているのは、そのうち7.2人。

7.2/76.64=0.09394572
つまり約9.4\%。

10\%に満たないわけです。
直感に反するなとおもった方がいるかもしれません。

%\Tree [ .1000人 [.有病者8人 真陽性7.2人 偽陰性.8人]%
%[ .無病者922人 真陰性922.56人 擬陽性69.44人] ]

\Tree [ .Win [ .Documents MyDocuments StartMenu]%
[ ProgramFiles CommonFiles Internet\\Explorer ] ]
\end{document}