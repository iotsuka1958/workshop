\documentclass[uplatex,jis2004,dvipdfmx,12pt]{jsarticle}
\usepackage[hiresbb]{graphicx}
\usepackage[deluxe]{otf}
\usepackage{amsmath}
\usepackage{longtable}%%%hyperrefより先に読むのがだいじらしい
\usepackage{okumacro}
%%%%%%%%%%%%%%%%%
\PassOptionsToPackage{dvipsnames,table,dvipdfmx}{xcolor}%tikzパッケージよりも前に読み込みます。
\usepackage[bww,arrow= red]{callouts}
\usepackage{tikz}
\usepackage{pxpgfmark} % remember picture を可能にする
\usepackage[yyyymmdd]{datetime}
\usepackage{array,colortbl,xcolor}
\usepackage{enumitem}%%[label=\textbf{\arabic*}]
\usepackage{arydshln}
\usepackage{amsmath}
\usepackage{amssymb}
\usepackage{niceframe}
\usepackage{multicol}
\usepackage{bxwareki}%%%和暦
\usepackage{mymacros}
%%%%%%%%%%%%%%%%%%%%%%
\usepackage{ascmac}
\usepackage{booktabs}%%%%%tabularの横線の改良\toprule\midrule\bottomrule
%%%%%%%%%%%%%%%%%%%%%%%%%%%
\usepackage{qtree}
%%%%%%%%%%%%%%%%%%%%%%%%%%
\begin{document}


\section{データ}
とある予算関係の会議でのことです。

「この事業ではこれまでどんな成果がありましたか」と
問われた課の職員がこう答えました。
「子供の目が輝き、笑みがこぼれました」

これは実に美しい話だし、
うそ偽りなくきらめく笑みがこぼれたとおもうのです。
これは究極のアウトカムかもしれませんが、
これで財布を握っている財政当局が予算を認めてくれるかどうかは、
また別のお話です。
というか即出直しレベル。




行政職員として、自分の思いを事業化して、
社会のためになにかしたいというところは、
わたくしどもみな共通かなとおもいますが、
たしかに思いは重要ですが、
思いだけでは仕事は前に進まない。
ここはどうしても客観的な根拠が必要な場面。


Evidence-Based Policy-Making,
make a policy based upon evidenceからできたことば、
「エビデンスに基づいて政策を立案する」ということです。

ということで、
前半はデータ・数字の話です。


\subsection{病気の確率}

わたくし、先日、ある病気の検査を受けました。

人口の1\%の人が、この病気にかかっていることが知られています。

有病率 $=$ 1\%

で、わたしの検査結果は陽性でした。

この検査、
この病気にかかっている人が受けると確率90\%で陽性。

この確率のことを「感度」といいます。日常的な感覚でわかりやすいことば。

また、
この病気にかかっていない人が受けると確率90\%で陰性。
この確率を「特異度」といいます。
こちらの用語はちょっとわかりにくいですね。

病気の場合、90\%が陽性になるということは、残りの10\%は病気を見逃されるというこ
と、
病気でない場合90\%で陰性になるということは、
残りの10\%は病気でないのに陽性になって心配することになります。

完璧な検査というものが理想ですが、なかなかそうはいきません。
完璧な検査は感度も特異度もともに100\%ですが、
通常これらはトレードオフの関係。
片方がよくなるともう一方は悪くなります。

きょくたんなことをいいます。
感度をあげたければ、かたっぱしから陽性にすればいい。
そうすれば見落としはなくなり感度は100\%に近づいていきますが、
病気でないのに陽性になる人がたくさん
でてくる、つまり特異度がさがってきます。

その逆もしかり。
特異度をあげたければ「疑わしきは罰せず」というか、
基本陰性にすれば特異度は100\%に近づきます。
そのかわり病気の人が見逃されてしまう。
感度が鈍くなります。
なかには手遅れになってしまう人もでてくるでしょう。

一般的に言えばそういうことになります。

わたしが受けた検査は、
感度90\%、特異度90\%ですから、
完璧ではないけれど、まあまあ優秀な検査。


さて、わたしが本当にこの病気にかかっている確率はどれくらいでしょうか。

「いやいやそれは90\%だろう」
「有病率は0.1\%なんだから0.1\%?」
「病気かそうじゃないかの2択だから50\%」
とか、けっこういろいろな意見が出てきておもしろいところ。
みなさんのカンでいうとどれくらいでしょうか。

指名はしませんのでリラックスしていただければ。

これ落ち着いて考えるとべつにむずかしくありません。
四則演算で、つまり小学生のレベルなのかなという問題。


1,000人いるとします。

そうすると有病率1\%ですから、
病気の人は10人。


感度は90\%なので、
陽性になるのは$10*.9=9$人。


いっぽう病気でない人は、
1000-10=990人。

特異度が90\%ですから、
陰性になるのは
990*.9=891人。

10\%の人が陽性になってしまうので
990*.1=99人が陽性。

そうすると
陽性になるのは、ふたつあわせて9人+99人=108人。

実際に病気にかかっているのは、そのうち9人。

$9/108=0.83333\dots$
つまり約8.3\%。
$8.\dot{3}\%$

10\%に満たないわけです。
これ、これまでの経験ですと、
意外だという感想を持つ人が多いところです。
皆さんはどうでしたでしょうか。
直感に反するなとおもった方がいるかもしれません。

%\Tree [ .1000人 [.有病者8人 真陽性7.2人 偽陰性.8人]%
%[ .無病者922人 真陰性922.56人 擬陽性69.44人] ]

\bigskip


\begin{tikzpicture}
[
    level 1/.style={sibling distance=80mm},
    level 2/.style={sibling distance=40mm},
]
	\node {1,000人}
		child {
		    node {\begin{tabular}{c}病気\\10人\end{tabular}}
		    child {node {%
\begin{tabular}{c}
$10*0.9=9人$\\(真陽性)
\end{tabular}
}}
		    child {node {$10*0.1=1人$}}
                 }
		child {
		    node {\begin{tabular}{c}病気でない\\990人\end{tabular}}
		    child {node {$990*.9=891人$hoge}}
		    child {node {%
\begin{tabular}{c}
$990*.1=99人$\\(疑陽性)
\end{tabular}%
}}
		};
\end{tikzpicture}



  \scalebox{.66}{%
\begin{tikzpicture}
[
    level 1/.style={sibling distance=80mm},
    level 2/.style={sibling distance=60mm},
]
	\node {1,000人}
		child {
        node {\begin{tabular}{c}病気\\10人\end{tabular}}
		    child {node {%
\begin{tabular}{c}
9人\\(真陽性)
\end{tabular}
}}
		    child {node {%
		    \begin{tabular}{c}
1人\\(偽陰性)
\end{tabular}}}
                 }
		child {
		    node {\begin{tabular}{c}病気でない\\990人\end{tabular}}
		    child {node {%
		    \begin{tabular}{c}
891人\\(真陰性)
\end{tabular}
		    }}
		    child {node {%
\begin{tabular}{c}
99人\\(偽陽性)
\end{tabular}%
}}
		};
\end{tikzpicture}%
}



あ、でわたし嘘つきました。
検査を受けたというのは嘘でした。お許しください。

医師国家試験ででた問題の数字を変えただけ


\subsection{Simpson's paradox}
もうひとつやってみます。



「家庭学習の習慣が重要だ」ということで、ある施策を展開しようとおもいます。

ある自治体で複数の小学校で共通学力テストをやったとします。
家庭学習の時間についてもあわせて調査しました。

わたしとしては、家庭学習の時間が長い子供ほど正答率もいいはずだとおもい
ます。

で、プロットしたらこうなりました。

ちょっと嫌な感じ。

なんか右肩下がり。
実際にこの傾向を直線で近似するとこんな感じ。

「テストの点をあげるには勉強しないほうがいい」ということになるのでしょう
か。


さきほど4つの学校で調査したといいました。
色分けしてみます。

あれあれ。いかがですか。

4つの学校ごとに見ると、勉強すればするほど上がっているみたい。
近似直線を書いてみます。

これどう考えたらいいでしょうか。

\begin{itemize}
 \item 全体でみれば、勉強するほど正答率が下がる
 \item ここに見れば、勉強するほど正答率が上がる
\end{itemize}
考察はみなさんにおまかせすることにしますが、
といっていいことになるはずです。
これは実際のデータではなく頭の体操ということですが、
現実にも十分起こりうる現象であるとおもっております。

全体の傾向とグループに分けた時の個々のグループの傾向が逆転する---
これしばしばありまして、
最近の例でいいますと、
イスラエルで新型コロナクチンの有効性についての調査があって、
これは結論だけですが、
全体的には67.5\%の有効性であり、
%インフルエンザなんかは50\%程度ということですので、
%なかなか優秀な数字だったわけですが、
年齢別でみると各層において80\%後半から90\%超とかなりの数字をたたき出した、
これもシンプソンのパラドックスの例です。

ということで、数字といいますかデータをだいじにして、
おおげさにいいますと教育のため、千葉県のために、
みなさんの思いを施策として事業として展開していただければとおもいます。


\section{文書}
ただ実際に展開するためには案を起こす必要がございます。
案を起こすとは、漢語で言えば起案するということになります。
そして上司のハンコをもらわないといけない。

そこで問題になってくるのが文書のつくり。
電子がデフォルトになり、実際に紙に印刷するかどうかは別のお話でありますが、
ここからは文書・決裁について考えていることを申し上げます。

決裁
       \begin{itemize}
	\item 「起案の手引」矛盾$\longrightarrow${}混乱
	\item 電子決裁$\longrightarrow${}差し戻し$\longrightarrow${}萎縮
	\item 漢字・かな
	\item レイアウト
	\item 固有名詞・数字・日付・金額
	\item スピード感
   \end{itemize}

雑にいますと
決裁について、ちょっと身構えすぎなのではないかということです。
決裁はもっと日常的なあたりまえのことであるべきです。

\subsection{起案の手引}
この職場には
「起案の手引」というものがあります。
親切だとおもう半面、それ自体に矛盾があることや、
ひな形を機械的に当てはめることができない場面、
できなくはないが無理やり当てはめるとおかしなことになる場合が現実にあり、
いろいろと混乱しかねません。

起案の手引のとおりやれといわれ、
おかしいなと思いつつも
そのとおりにやると、手直しがはいることがあるはずです。
「でも手引にこうあるからこうした」といえば、
「杓子定規にはいかない、臨機応変にやれ」などといわれる。

これでは立つ瀬がありません。
「もう余計な神経はつかわない。とりあえず提出して直しの指示があれば、
「はいはい、仰せのとおりになおします」」となってしまうのがいちばんよくない。
本県教育のために
前向きな姿勢で臨んでいたはずの職員が投げやりで思考停止になって
しまうようなことあってはうまくありません。

\subsection{だいじなこと}
決裁で大事なのは、
本筋が何かということであるはずです。
それが、見かけ上のレイアウトや文字使いといったことに
矮小化されているのではないかと思うときがあります。

文字使いやレイアウトなどどうでもいいといっているわけではありません。
非常に重要な要素であることはもちろんですが、
まずは本筋がいちばんだいじ。
なぜこういう案を起こすのか、
根拠は何なのか、目的はなんなのか。
そこを押さえた上で、
文字使いや体裁というのが筋であるはず。

もちろん
固有名詞、数字、金額、日付、曜日、ここは
ぜったいまちがってはいけない。
ここは慎重に確認すべきところですが、
それ以外のこと、
例えば
ここは漢字なのかひらがななのかといったところは、
誤解を恐れずに言えばたいしたことではありません。
文字使いについては、
政策法務課が作成している「公文書作成の手引」どおりにやればいいのですが、
多少の
漏れがあってもたいした傷ではありません。
例えば「又は」「または」、
公用文では漢字の「又は」となっていますが、
そこをひらがなで書いたときに実害はありません。

千葉県では従来から「てびき」を名詞で使うときは
「手引」と表記することとされ、
その旨が「公用文作成の手引」にも書かれていましたが、
その冊子自体が「公用文作成の手びき」というタイトルだったのです。
おおいなる矛盾です。
今はそこは訂正され「公用文作成の手引」となりました。

総本山でさえ長年まちがったままだったのですから、
みなさんがそんなに萎縮する必要はない。

\subsection{避けたいこと}
誤字脱字は恥ずかしいとはおもいますが、
しかし、そこは気づいた人が指摘してあげればいいだけの話。
いちいち差し戻しなどするのは非能率的。
紙の決裁の時代はいちいち差し戻しなどせずに
鉛筆で直しを入れてどんどん上に回していました。
電子になっていちいち差し戻すというのはなんとかならないのかとおもいます。
テクニック上のことは詳しい人にお任せしたいとおもいますが、
気づいた人がコメントをいれて上に回すとか、
直接に直しをいれて上にあげるとかのほうが能率的です。

それから同じ人が何度も差し戻す、五月雨式もいただけません。

なにか鬼の首でも取ったように「ここはこうだああだ」といわれると、
萎縮してしまいます。

さらに、公用文の手引きでも特段の定めがないようなことにうるさいのも疑問で
す。
いちおうローカルな取り決めといいますか、
ローカルルールを作るのはありだとはおもいますが、
ルールを作るときはよくよくさまざまな場合を想定しておかないと、
現実の場面で破綻しかねません。
ごくごく基本的なとこだけ押させてあとは起案者に任せればいいはず。

ひとつの単語が2行に分かれることを、校正の用語で
「泣き別れ」というようです。
このいわゆる「泣き別れ」を指摘する人もいるのですが、
「泣き別れ」についてはまちがいではないので、
必ず修正すべきものではないというのが私の考えです。
指摘するのは簡単ですが、
この調整はかなりてまです。
一つを直すと別のところでおかしくなったりします。
公用文作成の手引でも記載はないことを指摘しておきます。


最悪なのは、
ここは半角右に寄せてとかいうレイアウトについての指示。
個人的な好みでものはいわないほうがいい。


wordを代表とするワープロは、
そもそもあまり細かな部分まで制御できるソフトウェアでは
ありません。
よく、なにかしようとすると文書のスタイルが崩れてしまうとか耳にしますが、
wordよいうソフトウェアにそこまで求めること自体がまちがっています。


実は、
私自身は文書の体裁、レイアウトやフォントなど病的に気になるたちなのですが、
それを職員に求めたことはありません。
じぶんで文書を作成するときに細部までこだわることと、
それを人に求めることは違います。

あまり好みでものをいわれると、
美的感覚のことであなたにものをいわれたくないという気持ちになりかねません。


起案とか決裁とかは特別なことではありません。
ごくごく日常的なこと。

本来役所では部署やときどきの業務にもよりますが、
日常的に起案します。
年間でいえば何十、百を超える起案をする人も珍しくありません。

起案がこわくなってはいけない。
基本的には決裁をとることにしておくほうが安全なのですが、
起案が怖くなって決裁せずにだまてんで何かするほうが危ういです。

あまり起案とか決裁とか身構えずにやるべきです。

決裁する側は、どこまでこだわるか、逆に言えば、
どこから先は目をつぶるかを見きわめるのかが問われているのだとおもいます。

\subsection{結論}
乱暴な言い方ですが、
文書のできは8割程度でじゅうぶんではないか。
8割まではかんたんにできても、
そこから先を9割9割5分まで持っていくには相当の時間がかります。

けっきょく、ほかの業務が押せ押せになって、
時間外勤務ということにもつながってくるはずです。
働き方改革だとか、時間外勤務の縮減などといわれても、


数字や金額、日時、固有名詞に過ちがなければ、
体裁がそこそこ整っていればいいのではないかと思うのです。






さきほど役所では決裁は日常的なことであり、
特別なことではないといいました。

決裁のスピード感です。
通常であれば1日2日で決裁がもどってきます。
4日も5日も決裁が返ってこないなどということはありえません。
それでは仕事が前に進みませんから。


最後に、
この掲示、みなさん目にしたことがあるとおもいます。
本館1階はいって正面の柱に掲示してあります。

この掲示、わたしの知る限り20数年前からあったのですが、
はじめて見た時からよく決裁がとおったなあという思いが禁じえません。

これはあきらかにねじれた文章で
「お弁当の引換場所は、メディア教育棟1階大ホール前で行っています」
「引き換え場所はおこなっています」は恥ずかしい。
「引き換え場所は大ホール前です」ないし「引き換えは大ホール前で行っています」
とするべきだと申し上げて終わります。



第2次世界大戦で壊滅の危機に瀕していた英国の宰相であった Sir Winston Churchill
のことばです。

\includegraphics[height=\textheight]{churchill_memo.jpg}

出だしだけ確認します。

To do our work, we all have to read a mass of papers.

 Nearly all of them are far too long.

 This wastes time, while energy has to be spent in looking for the essential points.



内部の文書と外に飛んでいく文書

内部の文書は、体裁にこだわりすぎない

よくわきで聞いていると、wordでどうしてもはじがうまくそろわないとか、
そうするとなにか裏技テクニックを駆使してうまくやる人がでてきて、
名人とかいわれたりするのですが、
そういう人はすばらしいとおもいますが、
あまりそういうことに時間をかけるのはむだ。

そもそもwordに、といいますかいわゆるワードプロセッサにそこまで求めるのは筋違い。
商業出版物、
一定のレベルを超えた文書を
wordで作ることなど通常ありません。



\end{document}



