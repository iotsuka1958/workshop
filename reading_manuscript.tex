\RequirePackage{plautopatch}
\documentclass[uplatex,jis2004,dvipdfmx,14pt]{jsarticle}
\usepackage[hiresbb]{graphicx}
\usepackage[deluxe]{otf}
\usepackage{amsmath}
\usepackage{longtable}%%%hyperrefより先に読むのがだいじらしい
\usepackage{okumacro}
%%%%%%%%%%%%%%%%%
\PassOptionsToPackage{dvipsnames,table,dvipdfmx}{xcolor}%tikzパッケージよりも前に読み込みます。
\usepackage[bww,arrow= red]{callouts}
\usepackage{tikz}
\usepackage{pxpgfmark} % remember picture を可能にする
\usepackage[yyyymmdd]{datetime}
\usepackage{array,colortbl,xcolor}
\usepackage{enumitem}%%[label=\textbf{\arabic*}]
\usepackage{arydshln}
\usepackage{amsmath}
\usepackage{amssymb}
\usepackage{niceframe}
\usepackage{multicol}
\usepackage{bxwareki}%%%和暦
%\usepackage{mymacros}
%%%%%%%%%%%%%%%%%%%%%%
\usepackage{ascmac}
\usepackage{booktabs}%%%%%tabularの横線の改良\toprule\midrule\bottomrule
%%%%%%%%%%%%%%%%%%%%%%%%%%%
\usepackage{qtree}
%%%%%%%%%%%%%%%%%%%%%%%%%%
\begin{document}
\section{でだし}
本日は「行政職としての仕事の進め方」というおそろしい演題をいただいております。

きょうの話は2つ。
%%%%%%%%%%%%%%%%next slide
1つは「データ」について、
2つ目は「文書」について。


\section{データ}
では、まず「データ」についてです。

%%%%%%%%%%%%%%%%%%%%next slide
とある予算関係の会議でのことです。

「この事業でどんな成果が見込まれますか」と
質問された職員がこう答えました。
%%%%%%%%%%%%%%%%%%next slide
「わたしの夢は子供を笑顔にしたい。
この事業によって子供を笑顔にしたい」

これは実に美しい話だし、
子供の笑顔---これは究極のアウトカムかもしれませんが、
これで予算がつくかどうかは、
また別のお話。
というか即出直しレベル。撃沈。

%%%%%%%%%%%%%%%%next slide
行政職員として、自分の思いを政策にまとめ事業に落とし込んで、
社会のためになにかしたいというところは、
みな同じだとおもいます。
%%%%%%%%%%%%%%%%%%%%%%next slide
たしかに思いは重要ですが、
思いだけでは仕事は前に進まない。
どうしても欠かせないのが客観的な根拠です。

%%%%%%%%%%%%%%%%%%%%next slide
根拠は「法律・規則」のこともあれば、
知事の選挙公約かもしれません。
そういう場合は、予算の心配はありません。
法律が根拠であれば予算はつけざるを得ないし、
知事の公約であれば、心配しなくても、
「予算増額するからもっとやれ」といわれることもある。
そうでない場合は、根拠となるデータが必要です。

これから
「データに強くなろう」ということで、ケーススタディを3つやります。

\newpage
\subsection{Case Study 1}
最初の事例です。

あなたはある病気の検査を受けました。

検査結果は陽性でした。

人口の1\%の人が、この病気にかかっていることが知られています。
有病率 $=$ 1\%

この検査の性能ですが、
病気にかかっている人が受けると確率90\%で陽性。

この確率のことを「感度」といいます。

また、
この病気にかかっていない人が受けると確率90\%で陰性。
この確率を「特異度」といいます。



完璧な検査は感度も特異度もともに100\%ですが、
そんな検査はありません。
やっかいなのはトレードオフの関係だということ。
片方がよくなるともう一方は悪くなります。

感度をあげたければ、かたっぱしから陽性にすればいい。
そうすれば見落としはなくなり感度は100\%に近づいていきますが、
病気でないのに陽性になる人がたくさん
でてくる、つまり特異度がさがってきます。

その逆もしかり。
特異度をあげたければ「疑わしきは罰せず」というか、
基本陰性にすれば特異度は100\%に近づきます。
そのかわり病気の人が見逃されてしまう。
感度が鈍くなります。


この検査は、
感度・特異度ともに90\%。


さて、検査で陽性といわれたあなたが
本当にこの病気にかかっている確率はどれくらいでしょうか。

キョトンとする人がでてきます。
「いやいやそれは90\%だろう」
「有病率は1\%なんだから1\%?」
「病気かそうじゃないかの2択だから50\%」
とか、けっこういろいろな意見が出てきます。

ここはカンでやるところではありません。
実際に計算してみます。


何人でもいいいのですが、
1,000人がこの検査を受けたとします。


そもそも有病率1\%ですから、
病気の人は10人。

残りの99\%つまり990人は病気にかかっていません。

感度は90\%なので、
病気の人で陽性になるのは10人の90\%で9人。
病気だと正しく判定されたので
true positiveといいます。
陽性はピンクにしておきます。

病気なのに、
1人は誤って陰性になってしまいます。
誤って陰性なのでfalse negativeといいます。

病気でない人990人について考えます。

特異度が90\%ですから、
陰性になるのは
990かける9は891人。
正しく陰性判定ですから、
true negativeです。

病気ではないのに、
990人のうち10\%、つまり99人は陽性になってしまいます。
誤って陽性になるのでfalse positiveです。
陽性なのでピンクです。

そうすると
陽性なのは、ピンクのところで。
9人と99人、合わせて108人。

実際に病気にかかっているのは、そのうち9人。

$9/108=0.083333\dots$
つまり約$8.\dot{3}\%$

10\%に満たないことになります。

いかがでしょうか。
この問題、
小学校レベルの四則演算で解けたことに注目してください。

\newpage
\subsection{Case Study 2}

2つ目の事例です。

第2次世界大戦中のアメリカでの実話です。


撃墜される飛行機が問題になっていました。
基地に戻ってきた飛行機のデータをもとに、
機体をどう補強すべきかが課題となりました。

赤い点は銃撃を受けた箇所。
どこを補強すればいいかでしょう。

赤い点が密集している箇所を補強したくなるかもしれません。
しかし、実際にアメリカが行った補強は、赤い点が存在しない箇所でした。


なぜでしょう。
アメリカの考えはこうです。


赤い点がないところを銃撃された飛行機もあるはず。
ところがそこを銃撃されながら生還した飛行機はない。
つまり赤い点がないところこそウィークポイントではないのか。
いっぽうで、
赤い点がある箇所は攻撃されてもなんとか生還できたというのです。

手元にあるのは生き残った飛行機のデータだけであり、
撃墜された飛行機のデータが欠落していることを考慮して、
判断が行われたということです。

データに偏りがある場合、
欠落したデータがある場合、
慎重に判断する必要があるということです。

卑近な例でいうと、
「合格体験記」を鵜呑みにして失敗するという
悲しい例がままあります。

\newpage
\subsection{CaseStudy 3---Simpson's paradox}
3つ目は「教育」の話題。

これは架空の話です。
ある自治体で共通学力テストを実施しました。
試験勉強した時間についてもあわせて調査しました。


さて、勉強時間と正答率にそもそも関係があるのか。
あるとしたらどんな関係があるだろう。

そんなに単純な話ではないだろうが、
まあ勉強するほうが正答率はいい傾向があるのではないか。


x軸が学習時間、y軸が正答率。
点は生徒一人ひとり。

ちょっと嫌な感じ。

なんか右肩下がり。
実際に直線で近似するとこんな感じ。

「学力向上には勉強しないほうがいい」ということになるのでしょうか。
それともテスト自体に妥当性がなかったのでしょうか。

学校ごとに
色分けしてみます。

いかがですか。

4つの学校ごとに見ると、勉強すればするほど上がっているみたい。
近似直線を書いてみます。
明らかに右肩上がり。

整理すると
\begin{itemize}[itemsep=5pt]
 \item 全体でみれば、勉強するほど正答率が下がる
 \item 個々に見れば、勉強するほど正答率が上がる
\end{itemize}

勉強すれば正答率があがるというほうが
直感と合致しているのですが、
ただそうすると今度は学校間の格差というやっかいな問題がたちはだかってきま
す。

学校によって教員のレベルに差があるのかとか、
いやいやこの学校は富裕層が住んでいる学区なんだとか、
論点はいろいろありそうです。


一般論でいうと、
全体の傾向と個々のグループの傾向が逆転する---
これSimpsonのパラドックスと呼ばれていますが、
実際にしばしばあることです。

最近の例でいいますと、
イスラエルで新型コロナクチンの有効性についての調査があって、
全体的には67.5\%の有効性であり、
年齢別でみると各層において80\%後半から90\%超とかなりの数字をたたき出した、
ということもありました。


3つの事例をあげました。

データに基づき根拠を持って仕事を進めよう。
そのためにデータや数字に強くなろうということを申し上げました。

\newpage

\section{文書}

では、きょうの2点目「文書」についてです。

\subsection{Churchillのメモ}

いまご覧いただいているのは、
Winston Churchill。

1940年、
壊滅の危機にあった英国の首相となったChurchillが
部下に送ったメモがあります。

メモのタイトルは「Brevity(簡潔)」。

これが実際の公文書。わずか1枚です。

メモの冒頭を見ると

\begin{quote}
我々が仕事を遂行するためには大量の文書を読まねばならない。
そのほとんどすべてがあまりに長すぎる。
これは時間の無駄だし、要点を見つけるのに苦労する。

皆さんににお願いしたい。
報告書を短くしてもらいたい。
\end{quote}


%%%%%%%%%%%%%%%%%%%%%%%%%%

80年の時を経て、今なお
我々の仕事にもそのままあてはまります。

文書作成でいちばんだいじなのが「簡潔に」ということです。

わたしが考える簡潔とは、
必要な要素はもれなく盛り込んであり、
かつ不要な要素はひとつもまぎれこんでいない
ということです。

歯切れよくいいきってほしい。

もってまわったいいまわしはやめましょう。
これは極端だと思うかもしれませんが、
現実にこの手の文はよくあります。


簡潔な文書は本質が押さえてあり、歯切れがよく、
要点が浮かび上がる。


要点がはっきりしていれば、
迅速な意思決定につながります。
スピード感。

\subsection{8割でよし}

ここで、もう一つ重要だと思っていることを申し上げます。
「完成度の追求」---ではありません。
その逆。
「完成度の追求」はやめましょう。

これでいいと思ったのに、
あとで見直すとあれこれ気になってくる。
こうしたほうがもっとわかりやすいのではないか、とかきりがないのです。
いつまでたっても終わりません。

芸術家が魂を削って作品を作成するのとはちがいます。
限られた時間の中で事務を進める我々は費用対効果を考えないといけない。

少々、乱暴ですが、文書のできは8割でよしとしましょう。


ただ、
数字、金額、日付、所在地、固有名詞など、
これだけはまちがえないように慎重にお願いします。

\newpage
\section{落穂拾い}

さて、落ち穂拾いです。

\subsection{落穂ひろい1 体裁}

「体裁」について考えていることを申し上げます。


体裁については、
「公用文作成の手引」に従うことになりますが、
クイズ形式で。

\subsubsection{または}
べつにどちらでも実害はありませんが、
手引では漢字の「又は」と書くことになっています。


\subsubsection{子供}
「こども」ですが、手引では2文字とも漢字で「子供」と表記することとされています。
現実には徹底しておらず、
県の総合計画では「ども」はひらがなです。
手引をだしている本家の知事部局も手引に従っていないことになります。
現行の教育振興基本計画は漢字の「子供」、
教育庁は手引どおり。

手引どおりなら漢字2文字でも、子サポはどこまでいっても
ひらがなの「ども」。
2つの表記が混在せざるを得ないのです。

\subsubsection{コンピュータ}
「コンピュータ」それとも「コンピューター」
県の手引ではどちらでもいい。
学習指導要領長音符号のない「コンピュータ」。

国にあわせて「コンピュータ」とすればよさそうです。
ところが、同じ---terでも、「総合教育センター」は長音符号つき。
長音符号のあるなしが混在してきます。

\subsubsection{手引}
最後に「手引」はどうでしょう。

県ではさまざまな手引を出していますが、どちらもあります。

「公用文作成の手引」は漢字2文字。
でも、上のタブはひらがなの「き」がはいってます。
これはご愛嬌ということで。

表紙、
「手びき」がひらがな2文字でまた別のパターン。
これ昔の版でいまは直っていますが、
かつては手引きが手引きに従っていなかった。
細かく見ればいろいろあるということ。


\subsubsection{体裁のまとめ}
手引に従いましょう。これが大原則。

ただ、今見たように、
実際にはいろいろあるので、
文字遣いとかレイアウトについては
ほどほどでいいと思っています。
体裁に過敏になって、
数字をまちがったり、
業務が停滞することは避けないといけません。

\newpage
\subsection{落穂拾い2 ローカルルール}
落穂拾いの2番目「ローカルルール」について。

ひとつの所属から発出される文書の体裁をすみずみまで統一するという
考え方があります。

この点についてのわたしの考えはこうです。

\begin{itemize}
\item ルールが増えれば担当者のコストが増えること(神経を使うことの増加)
\item ルールがすべての場合を想定していないために、現実の場面で破綻することがあること
\item 複数の文書を横並びで見る立場の人は気になるかもしれないが、
実際に文書を受け取る側は1つの文書しか目にしないので、その中で統一されていれば
気にしないこと
\item ある所属で統一しても、県にはいろいろな組織があり、どんなにがんばっても
他の所属との統一感ははかれないこと
 \item 細かく決めると、職員があまりものを考えなくなる。各自が創意工夫し
      てわかりやすい文書を作ろうとする気をそぐことになりがちです。
\item ときに「個人的な趣味」と思われる謎ルールがあること
\end{itemize}

これらのことから、
ローカルルールを作るのであれば、
費用対効果を熟慮したうえで、精緻なルールにする必要があると考えています。

\newpage

\section{最後に}
本日は、
「根拠」をもって「簡潔な文書」を心がけ、「要点を明確」にいして
仕事を進めたいという話をしました。

\subsection{弁当についての掲示}
冒頭、
ある人が
「私の夢は子供を笑顔にしたい」と発言したといいました。
話しことばではよくあることですが、書き言葉では直したくなります。

「私の夢は子供を笑顔にすることだ」とか、
「夢」を捨てて「私は子供を笑顔にしたい」とでもすればいいのでしょう。

では、この掲示を見てください。

100\%意味は通じます。まったく実害はありません。

ねじれてはいますが、
何十年も前から千葉県総合教育センター
本館正面玄関を入ったところに堂々と掲示されています。

みなさん「おおらかな気持ちで仕事をしよう」
と申し上げて、話を終わります。

\end{document}

