\documentclass[uplatex,jis2004,dvipdfmx,12pt]{jsarticle}
\usepackage[hiresbb]{graphicx}
\usepackage[deluxe]{otf}
\usepackage{amsmath}
\usepackage{longtable}%%%hyperrefより先に読むのがだいじらしい
\usepackage{okumacro}
%%%%%%%%%%%%%%%%%
\PassOptionsToPackage{dvipsnames,table,dvipdfmx}{xcolor}%tikzパッケージよりも前に読み込みます。
\usepackage[bww,arrow= red]{callouts}
\usepackage{tikz}
\usepackage{pxpgfmark} % remember picture を可能にする
\usepackage[yyyymmdd]{datetime}
\usepackage{array,colortbl,xcolor}
\usepackage{enumitem}%%[label=\textbf{\arabic*}]
\usepackage{arydshln}
\usepackage{amsmath}
\usepackage{amssymb}
\usepackage{niceframe}
\usepackage{multicol}
\usepackage{bxwareki}%%%和暦
\usepackage{mymacros}
%%%%%%%%%%%%%%%%%%%%%%
\usepackage{ascmac}
\usepackage{booktabs}%%%%%tabularの横線の改良\toprule\midrule\bottomrule
%%%%%%%%%%%%%%%%%%%%%%%%%%%
\begin{document}
本日は、限られた時間ですので、大きく2つ話をさせてください。

1つは「数字・データ」の話。
もうひとつは「文書」の話です。

では、「数字・データ」から。

とある予算関係の会議でのことです。

「この事業ではこれまでどんな成果がありましたか。今後はどんな成果が見込ま
れますか」と問われた課の職員がこう答えました。
「子供の目が輝き、笑顔がこぼれました。この事業によりもっと明るい笑顔が期
待できます」

これは実に美しい話だし、
嘘偽りなくきらめく笑顔がこぼれたとおもうのですが、
これで財布を握っている財政当局が予算を認めてくれるかどうかは、
また別のお話です。

ここはどうしても数字の裏付けが必要な場面です。

Evidence-Based Policy-Making,
make a policy based upon evidenceからできたことば、
「エビデンスに基づいて政策を立案する」ということで、
エビデンスは必ずしも数字である必要はないわけですが、

\end{document}