\documentclass[uplatex,jis2004,dvipdfmx,12pt]{jsarticle}
\usepackage[hiresbb]{graphicx}
\usepackage[deluxe]{otf}
\usepackage{amsmath}
\usepackage{longtable}%%%hyperrefより先に読むのがだいじらしい
\usepackage{okumacro}
%%%%%%%%%%%%%%%%%
\PassOptionsToPackage{dvipsnames,table,dvipdfmx}{xcolor}%tikzパッケージよりも前に読み込みます。
\usepackage[bww,arrow= red]{callouts}
\usepackage{tikz}
\usepackage{pxpgfmark} % remember picture を可能にする
\usepackage[yyyymmdd]{datetime}
\usepackage{array,colortbl,xcolor}
\usepackage{enumitem}%%[label=\textbf{\arabic*}]
\usepackage{arydshln}
\usepackage{amsmath}
\usepackage{amssymb}
\usepackage{niceframe}
\usepackage{multicol}
\usepackage{bxwareki}%%%和暦
%\usepackage{mymacros}
%%%%%%%%%%%%%%%%%%%%%%
\usepackage{ascmac}
\usepackage{booktabs}%%%%%tabularの横線の改良\toprule\midrule\bottomrule
%%%%%%%%%%%%%%%%%%%%%%%%%%%
\usepackage{qtree}
%%%%%%%%%%%%%%%%%%%%%%%%%%
\begin{document}


\section{データ}
とある予算関係の会議でのことです。

「この事業ではこれまでどんな成果がありましたか」と
問われた課の職員がこう答えました。
「子供の目が輝き、笑みがこぼれました」

これは実に美しい話だし、
うそ偽りなくきらめく笑みがこぼれたとおもうのです。
これは究極のアウトカムかもしれませんが、
これで財布を握っている財政当局が予算を認めてくれるかどうかは、
また別のお話です。
というか即出直しレベル。




行政職員として、自分の思いを事業化して、
社会のためになにかしたいというところは、
わたくしどもみな共通かなとおもいますが、
たしかに思いは重要ですが、
思いだけでは仕事は前に進まない。
ここはどうしても客観的な根拠が必要な場面。


Evidence-Based Policy-Making,
make a policy based upon evidenceからできたことば、
「エビデンスに基づいて政策を立案する」ということです。

ということで、
前半はデータ・数字の話です。


\subsection{病気の確率}

わたくし、先日、ある病気の検査を受けました。

人口の1\%の人が、この病気にかかっていることが知られています。

有病率 $=$ 1\%

で、わたしの検査結果は陽性でした。

この検査、
この病気にかかっている人が受けると確率90\%で陽性。

この確率のことを「感度」といいます。日常的な感覚でわかりやすいことば。

また、
この病気にかかっていない人が受けると確率90\%で陰性。
この確率を「特異度」といいます。
こちらの用語はちょっとわかりにくいですね。

病気の場合、90\%が陽性になるということは、残りの10\%は病気を見逃されるというこ
と、
病気でない場合90\%で陰性になるということは、
残りの10\%は病気でないのに陽性になって心配することになります。

完璧な検査というものが理想ですが、なかなかそうはいきません。
完璧な検査は感度も特異度もともに100\%ですが、
通常これらはトレードオフの関係。
片方がよくなるともう一方は悪くなります。

きょくたんなことをいいます。
感度をあげたければ、かたっぱしから陽性にすればいい。
そうすれば見落としはなくなり感度は100\%に近づいていきますが、
病気でないのに陽性になる人がたくさん
でてくる、つまり特異度がさがってきます。

その逆もしかり。
特異度をあげたければ「疑わしきは罰せず」というか、
基本陰性にすれば特異度は100\%に近づきます。
そのかわり病気の人が見逃されてしまう。
感度が鈍くなります。
なかには手遅れになってしまう人もでてくるでしょう。

一般的に言えばそういうことになります。

わたしが受けた検査は、
感度90\%、特異度90\%ですから、
完璧ではないけれど、まあまあ優秀な検査。


さて、わたしが本当にこの病気にかかっている確率はどれくらいでしょうか。

「いやいやそれは90\%だろう」
「有病率は0.1\%なんだから0.1\%?」
「病気かそうじゃないかの2択だから50\%」
とか、けっこういろいろな意見が出てきておもしろいところ。
みなさんのカンでいうとどれくらいでしょうか。

指名はしませんのでリラックスしていただければ。

これ落ち着いて考えるとべつにむずかしくありません。
四則演算で、つまり小学生のレベルなのかなという問題。


1,000人いるとします。

そうすると有病率1\%ですから、
病気の人は10人。


感度は90\%なので、
陽性になるのは$10*.9=9$人。


いっぽう病気でない人は、
1000-10=990人。

特異度が90\%ですから、
陰性になるのは
990*.9=891人。

10\%の人が陽性になってしまうので
990*.1=99人が陽性。

そうすると
陽性になるのは、ふたつあわせて9人+99人=108人。

実際に病気にかかっているのは、そのうち9人。

$9/108=0.83333\dots$
つまり約8.3\%。
$8.\dot{3}\%$

10\%に満たないわけです。
これ、これまでの経験ですと、
意外だという感想を持つ人が多いところです。
皆さんはどうでしたでしょうか。
直感に反するなとおもった方がいるかもしれません。


%%%%%%%%%%%%%%%%%%
\if0
%%%%%%%%%%%%%%%%%
%\Tree [ .1000人 [.有病者8人 真陽性7.2人 偽陰性.8人]%j
%[ .無病者922人 真陰性922.56人 擬陽性69.44人] ]

\bigskip


\begin{tikzpicture}
[
    level 1/.style={sibling distance=80mm},
    level 2/.style={sibling distance=40mm},
]
	\node {1,000人}
		child {
		    node {\begin{tabular}{c}病気\\10人\end{tabular}}
		    child {node {%
\begin{tabular}{c}
$10*0.9=9人$\\(真陽性)
\end{tabular}
}}
		    child {node {$10*0.1=1人$}}
                 }
		child {
		    node {\begin{tabular}{c}病気でない\\990人\end{tabular}}
		    child {node {$990*.9=891人$hoge}}
		    child {node {%
\begin{tabular}{c}
$990*.1=99人$\\(疑陽性)
\end{tabular}%
}}
		};
\end{tikzpicture}


\begin{tikzpicture}
    \tikzset{block/.style={rectangle, fill=cyan!10, text width=2cm, text centered, rounded corners, minimum height=1.5cm}};
    \node[block] {population} [level distance=3cm, level 1/.style={sibling distance=80mm}, level 2/.style={sibling distance=40mm}, edge from parent/.style={->,draw}]
        child{ node[block] {sick}
 }
        child{ node[block] {not sick}
              child{ node[block]{piyo} }
              child{ node[block]{bar} }
 };
 \end{tikzpicture}

\begin{tikzpicture}
    \tikzset{block/.style={rectangle, fill=cyan!10, text width=2cm, text centered, rounded corners, minimum height=1.5cm}};
    \node[block] {population} [level distance=3cm, level 1/.style={sibling distance=80mm}, level 2/.style={sibling distance=40mm}, edge from parent/.style={->,draw}]
        child{ node[block] {sick}
              child{ node[block]{hoge} }
              child{ node[block]{fuga} }
 }
        child{ node[block] {not sick}
              child{ node[block]{piyo} }
              child{ node[block]{bar} }
 };
 \end{tikzpicture}


\begin{tikzpicture}
 \tikzset{block/.style={rectangle, fill=cyan!10, text width=22mm, text centered, rounded corners, minimum height=1.5cm}};
 \tikzset{another_block/.style={rectangle, fill=red!10, text width=22mm, text centered, rounded corners, minimum height=1.5cm}};
\draw [help lines] (-6,0) grid (6,6);%(0,0)から(10,4)までの"細線の方眼"
\node[block] (population) at (0,6) {population\\1,000};
\node[block] (sick) at (-3.5,4) {sick};
\node[block] (not_sick) at (3.5,4) {not sick};
\node[another_block] (true_positive) at (-5,) {true positive};
\node[block] (false_negative) at (-2,1) {false negative};
\node[block] (true_negative) at (2,1) {true negative};
\node[another_block] (false_positive) at (5,1) {false positive};
\draw [->] (population) -- node[auto=right] {$1\%$} (sick); 
\draw [->] (population) -- node[auto=left] {$99\%$} (not_sick); 
\draw [->] (sick) -- node[auto=right] {$90\%$} (true_positive); 
\draw [->] (sick) -- node[auto=left] {$10\%$} (false_negative); 
\draw [->] (not_sick) -- node[auto=right] {$90\%$} (true_negative); 
\draw [->] (not_sick) -- node[auto=left] {$10\%$} (false_positive); 
 \end{tikzpicture}

%%%%%%%%%%%%%%%
\fi
%%%%%%%%%%%%%%%


\subsection{Simpson's paradox}
もうひとつ頭の体操やってみます。



「家庭学習の習慣が重要だ」ということで、ある施策を展開しようとおもいます。

ある自治体で複数の小学校で共通学力テストをやったとします。
家庭学習の時間についてもあわせて調査しました。

わたしとしては、家庭学習の時間が長い子供ほど正答率もいいはずだとおもい
ます。

で、プロットしたらこうなりました。

x軸が学習時間、y軸が正答率。

ちょっと嫌な感じ。

なんか右肩下がり。
実際にこの傾向を直線で近似するとこんな感じ。

「テストの点をあげるには勉強しないほうがいい」ということになるのでしょうか。



さきほど4つの学校で調査したといいました。
色分けしてみます。

あれあれ。いかがですか。

4つの学校ごとに見ると、勉強すればするほど上がっているみたい。
近似直線を書いてみます。
明らかに右肩上がり。


\begin{itemize}
 \item 全体でみれば、勉強するほど正答率が下がる
 \item ここに見れば、勉強するほど正答率が上がる
\end{itemize}
これどう考えたらいいでしょうか。


考察はみなさんにおまかせすることにしますが、
全体の傾向と個々のグループの傾向が逆転する---
これしばしばありまして、
Simosonのパラドックスと呼ばれています。



最近の例でいいますと、
イスラエルで新型コロナクチンの有効性についての調査があって、
これは結論だけですが、
全体的には67.5\%の有効性であり、
%インフルエンザなんかは50\%程度ということですので、
%なかなか優秀な数字だったわけですが、
年齢別でみると各層において80\%後半から90\%超とかなりの数字をたたき出した、
ということもありました。

\newpage

\section{文書}

思いだけでは前に進まないといいました。
データをベースにしよういうことなのですが、
実際に展開するためには、もうひとつ、だいじなことがある。
案を起こす---つまり「起案」して、上司の承認をもらわないといけない。

そこで問題になってくるのが文書のつくり。
%電子がデフォルトになり、実際に紙に印刷するかどうかは別のお話でありますが、
ここからは文書・決裁について考えていることを申し上げます。


%%%%%%%%%%%%%%%%%%%%%%%%%%

\if0
決裁
       \begin{itemize}
	\item 「起案の手引」矛盾$\longrightarrow${}混乱
	\item 電子決裁$\longrightarrow${}差し戻し$\longrightarrow${}萎縮
	\item 漢字・かな
	\item レイアウト
	\item 固有名詞・数字・日付・金額
	\item スピード感
   \end{itemize}
\fi


役所で働いていれば、起案・決裁はごくごく日常的なこと、

いま、ご覧いただいているのは「公用文作成の手びき」
総務部政策法務課がだしてるもの。
ちょっと古いやつですが。
むかしは文書課といいました。


文書は、基本的にはこの手引に従うだけのことなのですが、
だいじなのは本質的なこと。
なぜこういう案を起こすのか、
根拠は何なのか、目的はなんなのか。

そして、
内容をできるだけわかりやすく伝えるためには、体裁、レイアウトがだいじ。

そして、スピード感がだいじ

で、それが一般論ですが、
いまこの職場で思うのは
決裁について少し身構えすぎているのではないか。



起案にあたって、
皆さん
本質的なことを押さえているとはおもいますが、
少しばかり
体裁に気をとられすぎなのではないか。
漢字かひらがなか、送り仮名はどうするか、
ここは何文字分空ける、何行空けるとか、
つまりレイアウトにばかり気を取られすぎではないか---という感想を持っています。

もうひとつ申し上げると、
少しスピード感に欠けているのではないか。
差し戻しが多すぎるのではないか。
起案してから決裁が下りるのに3日も4日もかかるのはいかがなものか。


これでは仕事は回らない。



\subsection{公用文作成の手引}

\subsubsection{起案の手引}
「公用文作成の手引」に重ねて、
この職場には
「起案の手引」というものがあります。


ローカルルールを作るのはありだとはおもいますが、
ルールを作るときはいろいろな場合を想定しておかないと、
現実の場面で破綻しかねません。

「起案の手引」、
実に親切だとおもう半面、それ自体に矛盾や誤りがあることや、
ひな形を機械的に当てはめるとおかしなことになる場合が現実にあり、
いろいろと混乱しかねないとおもっております。

起案の手引のとおりにやったにもかかわらず、
上席から直しがはいる---そういう経験をお持ちの方が
いるかもしれません。
「手引にこうあるからこうした」といえば、
「あれはあくまでひな形だから。臨機応変にやれ」などといわれる。

これ、トラップといいますか、立つ瀬がないといいますか、
あまりこういうことが続くと、
ものを考えなくなります。
前向きな姿勢でいたはずの職員が投げやりになり、
思考停止に陥ってしまうようなことあってはうまくありません。

\subsection{どうすりゃいいのさ}


ではどうすればいいのかということです。

起案者は
大枠で手引どおりに作成するよう心がける。

金額、日付、数字等の事実については細心の注意が求められますが、
文字使いとかレイアウトについては、
誤解を恐れずにいえばまちがえてもたいしたことはありません。

実害がないのです。
せいぜい同業者がなにかいうくらいのこと。

クイズです。

文字使いについて
、
例えば、「又は」「または」、「子供」か「子ども」か、
「手引」か「手引き」か。


公用文では漢字の「又は」。

一般の名詞の場合、「子供」は
漢字で表記。
でも、「子サポ」は条例で「子ども」となってますから子だけ漢字。



「てびき」をマニュアルの意味で使うときは
「手引」と表記することとされています。

ただしさきほどもいったように、
このあたりは、「子サポ」を除けば、
仮にまちがってもたいしたことではありません。

もう一度、これを見てください。
「公用文作成の手びき」、タイトルが「手びき」。
おおいなる矛盾ですが、
実はこれ以前のもの。
現在は改訂第7版ですが、
「公用文作成の手引」となりました。

総本山の「手引」でさえ
その表紙が長年まちがったままだったのですから、
みなさんがそんなに萎縮する必要はない。


起案者は細心の注意を払うべきですが、それでも漏れがあったとき、
どうするか。
そこは気づいた人が指摘してあげればいいだけの話。
いちいち差し戻しするのは非能率的。
紙の決裁の時代はいちいち差し戻しなどせずに
鉛筆で直しを入れてどんどん上に回していました。
電子になっていちいち差し戻すというのはなんとかならないのかとおもいます。
%テクニック上のことは詳しい人にお任せしたいとおもいますが、
気づいた人がコメントをいれて上に回すとか、
直接に直しをいれて上にあげるとかのほうが能率的です。

%それから同じ人が何度も差し戻す、五月雨式もいただけません。

%なにか鬼の首でも取ったように「ここはこうだああだ」といわれると、
%萎縮してしまいます。

%さらに、公用文の手引でも特段の定めがないようなことにうるさいのも疑問です。

%%%%%%%%%%%%%%%%%%%%%
\subsection{泣き別れ}
ひとつの単語が2行に分かれることを、校正の用語で
「泣き別れ」といいます。
このいわゆる「泣き別れ」を指摘する人もいるのですが、
「泣き別れ」についてはまちがいではないので、
必ず修正すべきものではないというのが私の考えです。
文章のちょっとした工夫でなんとかなるなら程度の話。
これ、指摘するのは簡単ですが、
この調整はかなりてまです。
一つを直すと別のところでおかしくなったりします。
こういうところを均等割り付け等のテクニックを駆使してなんとかするのは、
時間の無駄。

公用文作成の手引でも記載はないことを指摘しておきます。

「泣き別れ」を避けるのは、読みまちがいがあってはならない読み原稿のとき。、
具体的には議会の答弁書では、泣き別れを避けるのですが、
それ以外の文書で過度に意識するのはよくない。

「公用文作成の手引」でも、いわゆる「泣き別れ」についての言及はありません。

%%%%%%%%%%
\if
最悪なのは、
ここは半角右に寄せてとかいう指示。
個人的な好みでものはいわないほうがいい。


wordを代表とするワープロは、
そもそもあまり細かな部分まで制御できるソフトウェアでは
ありません。
なにかしようとすると文書のスタイルが崩れてしまうとか耳にしますが、
wordよいうソフトウェアにそこまで求めること自体がまちがっています。
\fi




\subsection{last but not least}
私自身は文書の体裁、レイアウトやフォントなどかなり気になるたちなのですが、
それを職員に求めたことはありません。
じぶんが細部までこだわることと、
それを人に求めることは違います。

乱暴な言い方ですが、
文書のできは8割程度でじゅうぶんではないか。
数字や金額、日時、固有名詞、場所に過ちがなく、
体裁がそこそこ整っていればいいのではないかと思うのです。
8割のできの文書を9割、9割5分まで持っていくには相当の労力。
いつまでたっても終わりません。



それよりスピード感。


最後に、
この掲示、みなさん目にしたことがあるとおもいます。
本館1階はいって正面の柱に掲示してあります。

この掲示、わたしの知る限り20数年前からあったのですが、
はじめて見た時からよく決裁がとおったなあという思いが禁じえません。

これはあきらかにねじれた文章で
「お弁当引換場所は、メディア教育棟1階大ホール前で行っています」
「引換場所はおこなっています」は恥ずかしい。
「引換場所は大ホール前です」ないし「引換は大ホール前で行っています」
とするべきだと申し上げて終わります。



第2次世界大戦で壊滅の危機に瀕していた英国の宰相であった Sir Winston Churchill
のことばです。

\IfFileExists{churchill_memo.jpg}{\includegraphics[height=\textheight]{churchill_memo.jpg}}{\relax}

出だしだけ確認します。

To do our work, we all have to read a mass of papers.

 Nearly all of them are far too long.

 This wastes time, while energy has to be spent in looking for the essential points.



内部の文書と外に飛んでいく文書

内部の文書は、体裁にこだわりすぎない




\end{document}


