\documentclass[uplatex,jis2004,dvipdfmx,12pt]{jsarticle}
\usepackage[hiresbb]{graphicx}
\usepackage[deluxe]{otf}
\usepackage{amsmath}
\usepackage{longtable}%%%hyperrefより先に読むのがだいじらしい
\usepackage{okumacro}
%%%%%%%%%%%%%%%%%
\PassOptionsToPackage{dvipsnames,table,dvipdfmx}{xcolor}%tikzパッケージよりも前に読み込みます。
\usepackage[bww,arrow= red]{callouts}
\usepackage{tikz}
\usepackage{pxpgfmark} % remember picture を可能にする
\usepackage[yyyymmdd]{datetime}
\usepackage{array,colortbl,xcolor}
\usepackage{enumitem}%%[label=\textbf{\arabic*}]
\usepackage{arydshln}
\usepackage{amsmath}
\usepackage{amssymb}
\usepackage{niceframe}
\usepackage{multicol}
\usepackage{bxwareki}%%%和暦
%\usepackage{mymacros}
%%%%%%%%%%%%%%%%%%%%%%
\usepackage{ascmac}
\usepackage{booktabs}%%%%%tabularの横線の改良\toprule\midrule\bottomrule
%%%%%%%%%%%%%%%%%%%%%%%%%%%
\usepackage{qtree}
%%%%%%%%%%%%%%%%%%%%%%%%%%
\begin{document}
\section{でだし}
「行政職員としてのあり方」などというおそろしい演題をいただきました。
わたしは学校でスタートをきった者であり、
行政職員としてはしょせん素人なのですが、
いたずらに長く籍を置いてきた中で考えていることを
2つ申し上げます。

1つは「根拠」について、
もう1つは「文書」についてです。


日本を支える人材を育成する---これが教員の喜びとすれば、
自らの思いを政策にまとめ個別の事業に落とし込み
推進していくのが行政職の仕事。
社会のために役立ちたいという点では共通しています。


\section{データ}
とある予算関係の会議でのことです。

「この事業ではこれまでどんな成果がありましたか」と
問われた課の職員がこう答えました。
「子供の目が輝き、笑みがこぼれました」

これは実に美しい話だし、
うそ偽りなくきらめく笑みがこぼれたとおもうのです。
これは究極のアウトカムかもしれませんが、
これで財布を握っている財政当局が予算を認めてくれるかどうかは、
また別のお話です。
というか即出直しレベル。




行政職員として、自分の思いを事業に落とし込んで、
社会のためになにかしたいというところは、
わたくしどもみな共通かなとおもいます。
たしかに思いは重要ですが、
思いだけでは仕事は前に進まない。
ここでどうしても欠かせないのが客観的な根拠。

ひとことでいえば
Evidence-Based Policy-Making,
「エビデンスに基づいて政策を立案する」ということです。

具体的にいうと、データとか数字に強くなろうということになります。

これからケーススタディを3つやります。

\subsection{Case Study 1}

ある病気の検査を受けました。

検査結果は陽性でした。

人口の1\%の人が、この病気にかかっていることが知られています。

有病率 $=$ 1\%

この検査、
この病気にかかっている人が受けると確率90\%で陽性。

この確率のことを「感度」といいます。日常的な感覚でわかりやすいことば。

また、
この病気にかかっていない人が受けると確率90\%で陰性。
この確率を「特異度」といいます。
こちらの用語はちょっとわかりにくい。

病気の場合、90\%が陽性になるということは、残りの10\%は病気を見逃されるというこ
と、
病気でない場合90\%で陰性になるということは、
残りの10\%は病気でないのに陽性になって心配することになります。

完璧な検査が理想ですが、なかなかそうはいきません。
完璧な検査は感度も特異度もともに100\%ですが、
通常これらはトレードオフの関係。
片方がよくなるともう一方は悪くなります。

きょくたんなことをいいます。
感度をあげたければ、かたっぱしから陽性にすればいい。
そうすれば見落としはなくなり感度は100\%に近づいていきますが、
病気でないのに陽性になる人がたくさん
でてくる、つまり特異度がさがってきます。

その逆もしかり。
特異度をあげたければ「疑わしきは罰せず」というか、
基本陰性にすれば特異度は100\%に近づきます。
そのかわり病気の人が見逃されてしまう。
感度が鈍くなります。
なかには手遅れになってしまう人もでてくるでしょう。

一般的に言えばそういうことになります。

この検査は、
感度・特異度ともに90\%ですから、
まあまあ優秀な検査。


さて、本当にこの病気にかかっている確率はどれくらいでしょうか。

「いやいやそれは90\%だろう」
「有病率は1\%なんだから1\%?」
「病気かそうじゃないかの2択だから50\%」
とか、けっこういろいろな意見が出てきます。

実際に計算してみます。

1,000人がこの検査を受けたとします。

そもそも有病率1\%ですから、
病気の人は10人。

残りの99\%つまり990人は病気にかかっていません。

感度は90\%なので、
病気の人で陽性になるのは$10*.9=9$人。
病気だと正しく判定されたので
true positiveといいます。

病気なのに1人は、
誤って陰性になってしまいます。
誤って陰性なのでfalse negativeといいます。

病気でない人990人にうつります。

特異度が90\%ですから、
陰性になるのは
990*.9=891人。
正しく判定されて陰性ですから、
true negativeです。

病気ではないのに990人のうち、
10\%つまり99人が陽性になってしまいます。
誤って陽性になるのでfalse positiveです。

そうすると
陽性になるのは、ピンクのところ。
あわせて9人+99人=108人。

実際に病気にかかっているのは、そのうち9人。

$9/108=0.83333\dots$
つまり約8.3\%。
$8.\dot{3}\%$

10\%に満たないことになります。


ここは



\subsection{Case Study 2}

第2次世界大戦中のアメリカでの実話。

敵から銃弾を受けながらも基地に戻ってきた飛行機のデータをもとに、
機体をどう補強しようかという話。

赤い点が銃撃を受けた箇所。
これをもとに機体のどこを補強すればいいでしょう。

赤い点が密集したところを補強したくなるのが人情。
しかし、実際にアメリカが補強したのは赤い点がないところ。

なぜでしょう。このデータは銃撃を受けながらも基地まで帰還した機体のデータ。
撃墜された飛行機についてはデータに含まれません。

アメリカの考えはこうです。

機体が銃撃を受ける箇所が均等であるとすれば、
赤い点がないところを銃撃された飛行機もあるはず。
ところがそこを銃撃されながら帰還した飛行機はない。
つまりそこを攻撃されると撃墜されてしまうと考えたのです。
赤い点がある箇所は攻撃されてもなんとか帰還できたというのです。





\subsection{CaseStudy 3---Simpson's paradox}
最後のケーススタディ。

ある自治体で小学校の共通学力テストをやったとします。
家庭学習の時間についてもあわせて調査しました。


さて、家庭での学習時間と正答率に関係があるのか。
あるとしたらどんな関係?

家庭学習の時間が長い子供ほど正答率もいいはずだとおもいます。

で、プロットしてみます。

x軸が学習時間、y軸が正答率。

ちょっと嫌な感じ。

なんか右肩下がり。
実際にこの傾向を直線で近似するとこんな感じ。

「学力向上には勉強しないほうがいい」ということになるのでしょうか。

さきほど4つの学校で調査したといいました。
色分けしてみます。

あれあれ。いかがですか。

4つの学校ごとに見ると、勉強すればするほど上がっているみたい。
近似直線を書いてみます。
明らかに右肩上がり。


\begin{itemize}
 \item 全体でみれば、勉強するほど正答率が下がる
 \item 個々に見れば、勉強するほど正答率が上がる
\end{itemize}
これどう考えたらいいでしょうか。

勉強すれば正答率があがるというほうが
直感と合致しているのですが、
ただそうすると今度は学校間の格差というやっかいな問題がたちはだかってきました。

今日は頭の体操であり、
考察はみなさんにおまかせすることにしますが、
全体の傾向と個々のグループの傾向が逆転する---
これ実際にしばしばありまして、
Simosonのパラドックスと呼ばれています。

最近の例でいいますと、
イスラエルで新型コロナクチンの有効性についての調査があって、
全体的には67.5\%の有効性であり、
%インフルエンザなんかは50\%程度ということですので、
%なかなか優秀な数字だったわけですが、
年齢別でみると各層において80\%後半から90\%超とかなりの数字をたたき出した、
ということもありました。




思いだけでは前に進まないということで、
データに基づき根拠を持って仕事を進めようということを申し上げました。
\newpage

\section{文書}


実際に仕事を進めるためには、もうひとつ、だいじなことがある。
案を起こす---つまり「起案」して、上司の承認をもらわないといけない。

そこで問題になってくるのが文書のつくり。
%電子がデフォルトになり、実際に紙に印刷するかどうかは別のお話でありますが、
ここからは文書・決裁について考えていることを申し上げます。

\subsection{Churchillのメモ}

いまご覧いただいているのは、
Winston Churchill。

彼は1940年壊滅の危機にあった英国の首相となりましたが、その年に彼
が
部下に送ったメモがあります。

タイトルはBrevityつまり「簡潔」というメモ。

これが実際の公文書。ぺら1枚です。

出だしだけみてみますと

\begin{quote}
我々が仕事を遂行するためには大量の文書を読まねばならない。
そのほとんどすべてがあまりに長すぎる。
これは時間の無駄だし、要点を見つけるのに苦労する。

皆さんににお願いしたい。
報告書を短くしてもらいたい。
\end{quote}

で、このあとには、
\begin{itemize}
 \item 歯切れよく言い切れ
       \begin{itemize}
       \item もってまわったいいまわしはやめろ
       \item 乱暴な表現でかまわない
       \end{itemize}
% \item 詳細な分析は付録にしろ
% \item 見出しだけのメモで口頭で補うのもあり

\end{itemize}
といった具体的な指示が続きます。


%%%%%%%%%%%%%%%%%%%%%%%%%%

80年余り前の英国の話ですが、
時を超えて
我々の仕事にもあてはまるはず。

第一は
本質的なことを押さえることにつきます。
何が本質かわかっているから、そこを端的に抽出できるのです。


なぜこの文書を作成しているのか、
目的は何か、
盛り込むべき事項は何か。

逆に言えばこれらを押さえることができれば、
要素を盛り込んでいく。



\if0
ただ、
決裁について少し身構えすぎているのではないか。
体裁に気をとられすぎなのではないか。
漢字かひらがなか、送り仮名はどうするか、
ここは何文字分空ける、何行空けるとか、
といったことにいささか気を取られすぎではないか---という感想を持っています。


もうひとつ申し上げると、
少しスピード感に欠けているのではないか。
差し戻しが多すぎるのではないか。
起案してから決裁が下りるのに3日も4日もかかるのはいかがなものか。


正書法という考え方があります。
ご覧いただいている公用文作成の手引は正書法に基づいている。

正書法とは「書き方」に関するルールにしたがっていこうというもの。

たとえば「このことば」はひらがなでなく漢字で書くとか、
読み仮名はこうふるとかいったルールです。
表記を統一しようというもの。

ご承知のように公用文作成の手引は正書法にのっとったもの。

いちおうわれわれは公務員としてこの手引きに従うことになります。

「いちおう」と歯切れが悪い言い方をしたのは、
正書法は基本従うべきだが、それを貫くのはなかなか困難です。

ひとつにはついうっかりということがある。

例えば「又は」はまたを漢字で書くこととされているが、
ついうっかりひらがなで書いてしまうといった場合。

うっかりではなく表記を統一できないこともあります。
学習指導要領では「コンピュータ」となっていますが、
われわれが文書を発出するときは「総合教育センタ」とはできません。

「子供」は漢字で書くこととされていますが、
「子どもと親のサポートセンター」は「ども」をひらがなでかかなくてはいけま
せん。

これでは仕事は回らない。

役所で働いていれば、起案・決裁はごくごく日常的なこと、

いま、ご覧いただいているのは「公用文作成の手びき」
総務部政策法務課がだしてるもの。


文書は、基本的にはこの手引に従うのことになりますが、
いちばんだいじなのは本質的なこと。
なぜこういう案を起こすのか、
根拠は何なのか、目的はなんなのか。

そして、
内容をできるだけわかりやすく伝えるためには、体裁、レイアウトがだいじ。

そして、スピード感がだいじ
\fi



\subsection{公用文作成の手引}

\subsubsection{起案の手引}
「公用文作成の手引」に重ねて、
この職場には
「起案の手引」というものがあります。

\if0
ローカルルールを作るのはありだとはおもいますが、
ルールを作るときはいろいろな場合を想定しておかないと、
現実の場面で破綻しかねません。
\fi

「起案の手引」、
実に親切だとおもう半面、それ自体に矛盾や誤りがあることや、
ひな形を機械的に当てはめるとおかしなことになる場合が現実にあり、
いろいろと混乱しかねないとおもっております。

起案の手引のとおりにやったにもかかわらず、
直しがはいる---そういう経験をお持ちの方が
いるかもしれません。
\if0
「手引にこうあるからこうした」といえば、
「あれはあくまでひな形だから。臨機応変にやれ」などといわれる。
\fi

これ、トラップといいますか、立つ瀬がないといいますか、
あまりこういうことが続くと、
ものを考えなくなります。
前向きな姿勢でいたはずの職員が投げやりになり、
思考停止に陥ってしまうようなことがあってはうまくありません。

\subsection{どうすりゃいいのさ}


ではどうすればいいのかということです。

起案者は
とうぜん手引どおりに作成するよう心がける。

金額、日付、数字等の事実については細心の注意が求められますが、
文字遣いとかレイアウトについては、
誤解を恐れずにいえばまちがえてもたいしたことはありません。

実害がないのです。
%せいぜい同業者がなにかいうくらいのこと。

クイズです。

文字遣いについて
、

公用文では漢字の「又は」。

一般の名詞の場合、「子供」は
漢字で表記。
でも、「子サポ」は条例で「子ども」となってますから「子」だけ漢字。



「てびき」をマニュアルの意味で使うときは
「手引」と表記することとされています。

ただしさきほどもいったように、
このあたりは、「子サポ」を除けば、
仮にまちがってもたいしたことではありません。

もう一度、これを見てください。
「公用文作成の手びき」、タイトルが「手びき」。
おおいなる矛盾ですが、
実はこれ昔のもの。
現在は改訂第7版ですが、
「公用文作成の手引」となりました。

総本山の「手引」でさえ
その表紙が長年まちがったままだったのですから、
みなさんがそんなに萎縮する必要はない。


起案者は細心の注意を払うべきですが、それでも漏れがあったとき、
どうするか。
そこは気づいた人が指摘してあげればいいだけの話。
いちいち差し戻しするのは非能率的。
紙の決裁の時代はいちいち差し戻しなどせずに
鉛筆で直しを入れてどんどん上に回していました。
本質にかかわることなら別ですが、
レイアウトや文字遣いのことで
いちいち起案者まで差し戻すのではなく、
%テクニック上のことは詳しい人にお任せしたいとおもいますが、
気づいた人がコメントをいれて上に回すとか、
直接に直しをいれて上にあげればいいとおもっておりますす。

%それから同じ人が何度も差し戻す、五月雨式もいただけません。

%なにか鬼の首でも取ったように「ここはこうだああだ」といわれると、
%萎縮してしまいます。

%さらに、公用文の手引でも特段の定めがないようなことにうるさいのも疑問です。

%%%%%%%%%%%%%%%%%%%%%
\if0
\subsection{泣き別れ}
ひとつの単語が2行に分かれることを、校正の用語で
「泣き別れ」といいます。
このいわゆる「泣き別れ」を指摘する人もいるのですが、
「泣き別れ」についてはまちがいではないので、
必ず修正すべきものではないというのが私の考えです。
文章のちょっとした工夫でなんとかなるなら程度の話。
なんでもそうですが、指摘するのは簡単でも、
調整はかなりてま。
一つを直すと別のところでおかしくなったりします。
こういうところを均等割り付けのテクニックなどを駆使してなんとかするのは、
時間の無駄。

公用文作成の手引でも記載はないことを指摘しておきます。

「泣き別れ」を避けるのは、読みまちがいがあってはならない読み原稿のとき。、
具体的には議会の答弁書では、泣き別れを避けるのですが、
それ以外の文書で過度に意識するのはよくない。

「公用文作成の手引」でも、いわゆる「泣き別れ」についての言及はありません。
\fi
%%%%%%%%%%


\subsection{last but not least}
私自身は文書の体裁、レイアウトやフォントなどかなり気になるたちなのですが、
%それを職員に求めたことはありません。
じぶんが細部までこだわることと、
それを人に求めることは違います。

乱暴な言い方ですが、
文書のできは8割程度でじゅうぶんではないか。
数字や金額、日時、固有名詞、場所に過ちがなく、
外形的には公用文作成の手引におおむね従っていればいいのではないかと思うのです。
8割のできの文書を9割、9割5分まで持っていくには相当の労力。
いつまでたっても終わりません。


そしてスピード感。決裁に何日もかかっていては仕事は前に進みません。


最後に、
次の文を
この掲示、みなさん目にしたことがあるとおもいます。
本館1階はいって正面の柱に掲示してあります。

この掲示、わたしの知る限り20数年前からあったのですが、
はじめて見た時からよく決裁がとおったなあという思いが禁じえません。

これはあきらかにねじれた文章で
「お弁当引換場所は、メディア教育棟1階大ホール前で行っています」
「引換場所はおこなっています」は恥ずかしい。
「引換場所は大ホール前です」ないし「引換は大ホール前で行っています」
とするべきです。

意味はきちんと伝わるが、通例まちがいとされる文が、
堂々と何年も掲げられているのです。
皆さん、あまり萎縮せずに文書づくりに望んでいただければとおもいます。



\end{document}


